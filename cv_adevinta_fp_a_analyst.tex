Sure, I can help you create a CV in LaTeX format based on the provided information. Below is a sample LaTeX code for a CV that highlights your experience, skills, and qualifications. I will structure it based on the job description and company details you've shared.

```latex
\documentclass[a4paper,10pt]{article}
\usepackage{geometry}
\usepackage{enumitem}
\usepackage{titlesec}
\usepackage{hyperref}
\usepackage{xcolor}

\geometry{left=1in, right=1in, top=1in, bottom=1in}

\setlength{\parindent}{0pt}
\setlength{\parskip}{1em}

\titleformat{\section}{\large\bfseries}{}{0em}{}[\titlerule]

\hypersetup{
    colorlinks=true,
    linkcolor=blue,
    urlcolor=blue
}

\begin{document}

\begin{center}
    {\LARGE Your Name} \\
    \vspace{5pt}
    \small
    \href{mailto:your.email@example.com}{your.email@example.com} \\
    \href{https://www.linkedin.com/in/yourprofile}{linkedin.com/in/yourprofile} \\
    \href{tel:+1234567890}{+1 234 567 890} \\
    Berlin, Germany \\
\end{center}

\section*{Professional Summary}
Data-driven and strategic FP\&A Analyst with 3-4+ years of experience in business analytics and finance. Proven track record of working on complex, cross-functional projects in tech-oriented environments. Highly skilled in financial planning, data analysis, and strategic business insights delivery. Fluent in English, with strong communication and interpersonal skills.

\section*{Professional Experience}

\textbf{FP\&A Analyst} \hfill 2023 - Present \\
Adevinta (mobile.de), Berlin, Germany
\begin{itemize}[leftmargin=0.5cm]
    \item Lead planning, forecasting, and reporting processes for various business sectors, collaborating with analytics and finance teams.
    \item Provide hands-on support in month-end close processes and business forecasting.
    \item Monitor business risks and opportunities, analyzing competitive trends for strategic decision-making.
    \item Support senior team members in financial projects and initiatives.
\end{itemize}

\section*{Education}

\textbf{Bachelor’s/Master’s Degree in Finance/Economics} \hfill University Name, Country \\
Year of Graduation

\section*{Skills}
\begin{itemize}[leftmargin=0.5cm]
    \item Financial Planning and Analysis
    \item Business Analytics
    \item SAP (S4H/R3), Looker, OneStream, SQL (Bonus)
    \item Strong presentation and communication skills
    \item English (Fluent), German (Plus)
\end{itemize}

\section*{Professional Development}
Participated in numerous development and training programs, leveraging the company's learning resources to continuously enhance professional skills and knowledge in AI technologies and data analysis.

\section*{Additional Information}
\textbf{Benefits Enjoyed at mobile.de:}
\begin{itemize}[leftmargin=0.5cm]
    \item 28 days of paid time off.
    \item €50 monthly transportation/mobility allowance.
    \item Development budget and access to coaching.
    \item Up to 20 weeks of fully paid parental leave.
    \item Option to work remotely up to 20 days per year.
    \item 24/7 Employee Assistance Programme.
\end{itemize}

\end{document}
```

Before using this template:
1. Replace placeholders like `your.email@example.com`, `yourprofile`, and `+1 234 567 890` with your actual details.
2. Update `University Name` and `Year of Graduation` with your actual educational background.
3. Add more sections if necessary, such as additional work experience, certifications, or languages if they apply to you. 

For detailed customization, you can adjust margins, font sizes, and other formatting options within the LaTeX document.